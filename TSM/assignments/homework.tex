%! Tex program = xelatex
\documentclass[UTF8]{ctexart}
% \documentclass{article}
\usepackage{fancyhdr}
\usepackage{extramarks}
\usepackage{amsmath}
\usepackage{amsthm}
\usepackage{amssymb}
\usepackage{amsfonts}
\usepackage{mathrsfs}
\usepackage{tikz}
\usepackage[plain]{algorithm}
\usepackage{algpseudocode}
\usepackage{pgfplots}
\usepackage{graphicx}
\usepackage{subfigure}
\usetikzlibrary{automata,positioning}

%
% Basic Document Settings
%

\topmargin=-0.45in
\evensidemargin=0in
\oddsidemargin=0in
\textwidth=6.5in
\textheight=9.0in
\headsep=0.25in

\linespread{1.1}

\pagestyle{fancy}
\lhead{\hmwkAuthorName}
\chead{\hmwkClass\ (\hmwkClassInstructor\ \hmwkClassTime): \hmwkTitle}
\rhead{\firstxmark}
\lfoot{\lastxmark}
\cfoot{\thepage}

\renewcommand\headrulewidth{0.4pt}
\renewcommand\footrulewidth{0.4pt}

\setlength\parindent{0pt}

%
% Create Problem Sections
%

\newcommand{\enterProblemHeader}[1]{
    \nobreak\extramarks{}{Chapter \arabic{#1} continued on next page\ldots}\nobreak{}
    \nobreak\extramarks{Chapter \arabic{#1} (continued)}{Chapter \arabic{#1} continued on next page\ldots}\nobreak{}
}

\newcommand{\exitProblemHeader}[1]{
    \nobreak\extramarks{Chapter \arabic{#1} (continued)}{Chapter \arabic{#1} continued on next page\ldots}\nobreak{}
    \stepcounter{#1}
    \nobreak\extramarks{Chapter \arabic{#1}}{}\nobreak{}
}

\setcounter{secnumdepth}{0}
\newcounter{partCounter}
\newcounter{homeworkProblemCounter}
\setcounter{homeworkProblemCounter}{1}
\nobreak\extramarks{Chapter \arabic{homeworkProblemCounter}}{}\nobreak{}

\newenvironment{homeworkProblem}{
    \section{Chapter \arabic{homeworkProblemCounter}}
    \setcounter{partCounter}{1}
    \enterProblemHeader{homeworkProblemCounter}
}{
    \exitProblemHeader{homeworkProblemCounter}
}

%
% Homework Details
%   - Title
%   - Due date
%   - Class
%   - Section/Time
%   - Instructor
%   - Author
%

\newcommand{\hmwkTitle}{前2章}
\newcommand{\hmwkDueDate}{2020年3月31日}
\newcommand{\hmwkClass}{热力学与统计物理作业}
\newcommand{\hmwkClassTime}{天文系}
\newcommand{\hmwkClassInstructor}{北京师范大学}
\newcommand{\hmwkAuthorName}{王正一 201711160128}

%
% Title Page
%

\title{
    \begin{center}
      \includegraphics[width=250pt]{BNU_name.png}
    \end{center}
    \vspace{2in}
    \textmd{\textbf{\hmwkClass:\ \hmwkTitle}}\\
    % \normalsize\vspace{0.1in}\small{交\ 于\ \hmwkDueDate\ 8:00am }\\
    \vspace{0.1in}\large{\textit{\hmwkClassInstructor\hmwkClassTime}}
    \vspace{3in}
}

\author{\textbf{\hmwkAuthorName}}
\date{}

\renewcommand{\part}[1]{\textbf{\large Part \Alph{partCounter}}\stepcounter{partCounter}\\}
\renewcommand{\QQQ}[1]{\textbf{\large Question \large{#1} \\}}

%
% Various Helper Commands
%

% Useful for algorithms
\newcommand{\alg}[1]{\textsc{\bfseries \footnotesize #1}}

% For derivatives
% \newcommand{\deriv}[1]{\frac{\mathrm{d}}{\mathrm{d}x} (#1)}
\newcommand{\derivt}[1]{\frac{\mathrm{d}}{\mathrm{d}t} (#1)}
\newcommand{\Derivt}[1]{\frac{\mathrm{D}}{\mathrm{D}t} (#1)}
\newcommand{\derivo}[2]{\frac{\mathrm{d} #1 }{\mathrm{d} #2}}
\newcommand{\derivp}[2]{\frac{\partial #1}{\partial #2}}
% For partial derivatives
\newcommand{\pderiv}[2]{\frac{\partial}{\partial #2} (#1)}

\newcommand{\bral}[1]{\left( #1 \right)}
\newcommand{\bram}[1]{\left[ #1 \right]}
\newcommand{\brab}[1]{\left{ #1 \right}}

% Integral dx
\newcommand{\dx}{\mathrm{d}x}
\newcommand{\Const}{\mathrm{C}}

% Alias for the Solution section header
\newcommand{\solution}{\textbf{\large Solution}}

% Probability commands: Expectation, Variance, Covariance, Bias
% \newommand{\E}{\mathrm{E}}
% \newcommand{\Var}{\mathrm{Var}}
% \newcommand{\Cov}{\mathrm{Cov}}
% \newcommand{\Bias}{\mathrm{Bias}}

\begin{document}

\maketitle

\pagebreak

\begin{homeworkProblem}
\QQQ{1.1}
    试求理想气体的体胀系数 \(\alpha\) ,压强系数\(\beta\),等温压缩系数\(\kappa_{T}\) 
    \\
    理想气体的物态方程:
    \[
      pV=nRT
    \]
    \[
      \alpha=\frac{1}{V}\bral{\derivp{V}{T}}_{T}=\frac{nR}{pV}=\frac{1}{T}
    \]
    \[
      \beta=\frac{1}{p}(\derivp{p}{T})_{V}=\frac{nR}{pV}=\frac{1}{T}
    \]
    \[
      \kappa_{T}=-\frac{1}{V}\bral{\derivp{V}{p}}_{T}=-\frac{1}{V}\bral{\frac{nRT}{p^{2}}}=\frac{1}{p}
    \]

\QQQ{1.2}
    证明任何一种具有两个独立参量\(T\)、\(p\)的物质,其物理状态方程可有实验测量的体胀系数\(\alpha\)、等温压缩系数\(\kappa_{T}\),根据下述积分求得:\(ln V = \int{\bral{\alpha dT -\kappa_{T} dp }}\),若\(\alpha=\frac{1}{T}\),\(\kappa_{T}=\frac{1}{p}\),试求物态方程。
    \[
      \begin{aligned}
        V&=V\bral{T,p}
        \\
        dV&=\bral{\derivp{V}{T}}_{p}dT+\bral{\derivp{V}{p}}_{V}dp
        \\
        \frac{dV}{V}&=\frac{1}{V}\bral{\derivp{V}{T}}_{p}dT+\frac{1}{V}\bral{\derivp{V}{p}}_{V}dp
        \\
        &=\alpha dT -\kappa_{T} dp 
        \\
      \end{aligned}
    \]
    积分得\(ln V = \int{\bral{\alpha dT -\kappa_{T} dp }}\)
    \\
    若\(\alpha=\frac{1}{T}\),\(\kappa_{T}=\frac{1}{p}\)
    \[
      ln V = \int{\bral{\frac{1}{T} dT -\frac{1}{p} dp }} = ln T -ln p + ln \Const
    \]
    即:
    \\
    \[
      pV=\Const T
    \]
    \[  \rightline{ \qedsymbol}\]

\QQQ{1.5}
  \[
    f\bral{\mathscr{T},L,T}=0
  \]
  得:
  \[
    \bral{\derivp{L}{T}}_{\mathscr{T}}\bral{\derivp{T}{\mathscr{T}}}_{L}\bral{\derivp{\mathscr{T}}{L}}_{T}=-1
  \]
  \[
    \begin{aligned}
      \bral{\derivp{\mathscr{T}}{T}}_{L}&=-\bral{\derivp{L}{T}}_{\mathscr{T}}\bral{\derivp{\mathscr{T}}{L}}_{T}
      \\
      &=-L \alpha \frac{A}{L}E=-\alpha AE
    \end{aligned}
  \]
  积分得:
  \[
    \Delta \mathscr{T}=-\alpha AE\bral{T_{2}-T_{1}}
  \]
\QQQ{1.11}
  \[
    p\bral{z}=p\bral{z+dz}+\rho \bral{z}gdz=p\bral{z}+\derivo{}{z}p\bral{z}dz+\rho \bral{z}gdz
  \]
  得
  \[
  \begin{aligned}
    \derivo{}{z}p\bral{z}dz&=-\rho \bral{z}gdz
    \\
    n=\frac{V}{V_{0}}&=\frac{m}{M_{r}}
    \\
    V_{0}&=\frac{M_{r}}{\rho \bral{z}}
    \\
    p(z)\frac{M_{r}}{\rho(z)}&=RT(z)
    \\
    \derivo{}{z}p(z)&=-\frac{M_{r}g}{RT(z)}p(z)
    \\
  \end{aligned}
  \]
  绝热过程:
  \[
    \frac{p^{\gamma-1}}{T^{\gamma}}=\Const
  \]
  \[
  \begin{aligned}
    (\derivp{T}{p})_{S}&=\frac{\gamma-1}{\gamma}\frac{T}{p}
    \\
    \derivo{}{z}T(z)&=(\derivp{T}{p})_{S}\derivo{}{z}p(z)
    \\
    \derivo{}{z}T(z)&=-\frac{\gamma-1}{\gamma}\frac{M_{r}g}{R}
  \end{aligned}
  \]
  得
  \[
    \derivo{T(z)}{z}=-10K/km
  \]
\QQQ{1.12}
  准静态绝热过程:
  \[
    c_{V}dT+pdV=0
  \]
  物态方程:
  \[
    pV=nRT
  \]
  \[
    \frac{c_{V}}{nR}\frac{dT}{T}+\frac{dV}{V}=0
  \]
  即:
  \[
    \frac{1}{\gamma-1}\frac{dT}{T}+\frac{dV}{V}=0
  \]
  \[ 
    ln F(T)=\int \frac{dT}{(\gamma-1)T}=lnV+ln\Const
  \]
  即:
  \[
    VF(T)=\Const
  \]
  \(\Const\)是常数
\\
\QQQ{1.17}
  \[
    \Delta S_{H_{2}O}=\int^{T_{2}}_{T_{1}}\frac{mc_{p}dT}{T} \approx 1304.6 J/K
  \]
  \[
    Q=mc_{p}\Delta T = 4.18 \times 10^{5}J
  \]
  \[
    \Delta S_{source}=-\frac{Q}{T} \approx -1120.6J/K
  \]
  \[
    \Delta S=\Delta S_{H_{2}O}+\Delta S_{source}=184J/K
  \]
\QQQ{1.19}
  设杆长为\(L\),杆与\(x\)轴重合
  \[
    dS_{0}=c_{p}dx \int_{T_{l}}^{\frac{1}{2}(T_{1}+T_{2})} \frac{dT}{T}=c_{p}dx ln \frac{\frac{T_{1}+T_{2}}{2}}{T_{2}+\frac{T_{1}-T_{2}}{L}x}
  \]
  \(c_{p}\)是定压比热容
  \[
    \begin{aligned}
      \Delta S=\int_{0}^{L} S_{0} &=c_{p}\int_{0}^{L}[ln \frac{T_{1}+T_{2}}{2} -ln(T_{2}+\frac{T_{1}-T_{2}}{2}x)]dx
      \\
      &=c_{p}L ln \frac{T_{1}+T_{2}}{2}-\frac{c_{p}L}{T_{1}-T_{2}}(T_{1}lnT_{1}-T_{2}lnT_{2}-T_{1}+T_{2})
      \\
      &=c_{p}(ln \frac{T_{1}+T_{2}}{2} -\frac{T_{1}lnT_{1}-T_{2}lnT_{2}}{T_{1}-T_{2}}+1)
    \end{aligned}
  \]

\QQQ{1.21}
  
  \[
    \Delta S=\Delta S_{物}+\Delta S_{机}+\Delta S_{源} \ge 0
  \]
  \[
    \Delta S_{机} \ge 0
  \]
  \[
    Q=Q^{\prime}+W
  \]
  \[
    \Delta S_{源}=\frac{Q^{\prime}}{T_{2}}=\frac{Q-W}{T_{2}}
  \]
  \[
    \Delta S \ge S_{1}-S_{2}+\frac{Q-W}{T_{2}}\ge 0
  \]
  取等号时:
  \[
    W_{max}=Q-T_{2}(S_{1}-S_{2})
  \]
\end{homeworkProblem}

\pagebreak

\begin{homeworkProblem}
\QQQ{extra1}
  \[
    \begin{aligned}
      c_{V}&=(\derivp{V}{T})_{V}=T(\derivp{S}{T})_{V}
      \\
      c_{p}&=(\derivp{H}{T})_{p}=T(\derivp{S}{T})_{p}
      \\
      c_{p}&-c_{V}=T(\derivp{S}{T})_{V}-T(\derivp{S}{T})_{p}
    \end{aligned}
  \]
    将函数\(S(T,V)\)转换为\(S(T,p)\):
  \[
    \begin{aligned}
      dS&=(\derivp{S}{T})_{V}dT+(\derivp{S}{V})_{T}dV
      \\
      &=(\derivp{S}{T})_{p}dT+(\derivp{S}{p})_{T}dp
    \end{aligned}
  \]
  \(dp=0\):
  \[
    \begin{aligned}
      (\derivp{S}{T})_{V}&=(\derivp{S}{T})_{p}-(\derivp{S}{V})_{T}(\derivp{V}{T})_{p}
      \\
      c_{p}-c_{V}&=T(\derivp{S}{T})_{p}-T[(\derivp{S}{T})_{p}-(\derivp{S}{V})_{T}(\derivp{V}{T})_{p}]
      \\
      &=T(\derivp{S}{V})_{T}(\derivp{V}{T})_{p}
      \\
      &=T(\derivp{p}{T})_{V}(\derivp{V}{T})_{p}
      \\
    \end{aligned}
  \]
  得\(pV=nRT\)或\(c_{p}-c_{V}=nR\)

\QQQ{2.2}
  \[
    \begin{aligned}
      p&=f(V)T
      \\
      (\derivp{p}{T})_{V}&=(\derivp{p}{T})_{V}
      \because (\derivp{S}{V})_{T}&=(\derivp{p}{T})_{V}
      \\
      (\derivp{U}{V})_{T}&=T(\derivp{S}{T})_{V}-p
      \\
      \because (\derivp{U}{V})_{T}&=T(\derivp{p}{T})_{V}-p
      \\
      \therefore (\derivp{U}{V})_{T}&=Tf(V)-p=0
    \end{aligned}
  \]
  \(\therefore U\)与\(V\)无关
  \[  \rightline{ \qedsymbol}\]

\QQQ{2.3}
  \text{(a)}
  \[
    dH=TdS+Vdp
  \]
  令\(dH=0\):
  \[
    (\derivp{S}{p})_{H}=-\frac{V}{T} \textless 0
  \]
  \text{(b)}
  \[
    dU=TdS-pdV
  \]
  令\(dU=0\):
  \[
    (\derivp{S}{p})_{U}=\frac{p}{T} \textgreater 0
  \]
  \[  \rightline{ \qedsymbol}\]
\QQQ{2.4}
  \[
    \begin{aligned}
      U(T,p)&=U[T,V(T,p)]
      \\
      (\derivp{U}{p})_{T}&=(\derivp{U}{V})_{T}(\derivp{V}{p})_{T}
      \\
      \therefore (\derivp{U}{V})_{T}&=(\derivp{V}{p})_{T}=0
    \end{aligned}
  \]
  \[ \rightline{ \qedsymbol}\]


\QQQ{2.7}
  \[
    \begin{aligned}
      pV&=f(T),U=U(T)
      \\
      \therefore (\derivp{U}{V})_{T}&=T(\derivp{p}{T})_{V}-p=0
      \\
      (\derivp{p}{T})_{V}&=\frac{1}{V}\derivo{f}{T}
      \\
      \therefore T\derivo{f}{T} &= f
    \end{aligned}
  \]
  积分得:\(lnf=lnT+lnC\)或\(pV=CT\)

\QQQ{2.14}
  \[
    \derivp{E}{t}=\sigma T^{4}R_{s}^{2}d\Omega
  \]
  得:
  \[
    T=(\frac{1.35 \times 10^{3}R_{se}^{2}}{\sigma R_{s}^{2}})^{\frac{1}{4}} \approx 5670K 
  \]
% \QQQ{2.16}
%   \[
%     \
%   \]
\QQQ{extra2}
  估算宇宙第一缕光产生时的宇宙年龄。
  \[
  \\
  \]
  我们把宇宙的膨胀看做一个绝热过程,根据热力学第一定律:
  \[
    dU+pdV=0
  \]
  其中\(U=(\rho_{m}+\rho_{r})V=\rho V \)
  \(\rho_{m}\)和\(\rho_{r}\)分别是物质密度和辐射密度,宇宙尺度因子是\(a(t)\)
  \[
    \therefore V \propto a^{3}(t)
  \]
  得:
  \[
    \derivo{}{t}(\rho a^{3})+p\derivo{}{t}(a^{3})=0
  \]
  物质粒子是非相对论性的:
  \[
    p=p_{r}=\frac{\rho_{r}}{3}
  \]
  得:
  \[
    \derivo{}{t}(\rho_{m} a^{3})+\frac{1}{a}\derivo{}{t}(\rho_{r} a^{4})=0
  \]
  假设物质是严格守恒的:
  \[
    \derivo{}{t}(\rho_{m} a^{3})=0,\frac{1}{a}\derivo{}{t}(\rho_{r} a^{4})=0
  \]
  得:
  \[
    \rho_{m}=\rho_{m_{0}}(\frac{a_{0}}{a})^{3}
  \]
  \[
    \rho_{r}=\rho_{r_{0}}(\frac{a_{0}}{a})^{4}
  \]
  其中\(\rho_{m_{0}}\)和\(\rho_{r_{0}}\)分别是现在宇宙的物质密度和辐射密度
  \[
    \rho_{r} \propto T^{4}
  \]
  得:
  \[
    T=T_{0}\frac{a_{0}}{a}
  \]
  形成中性氢需要降到的温度为\(T_{\alpha}\):
  \[
    \epsilon_{\alpha}=k_{B}T_{\alpha}
  \]
  \[
    H^{2}=\frac{8\pi G}{3}(\rho_{r}+\rho_{m})+\frac{\Lambda}{3}-\frac{k}{a^{2}}
  \]
  \(H\)是\(Hubble\)常数,对于\(\Lambda=0,k=0\)的\(\text{Einstein-de Sitter}\)宇宙,假设\(a(t_{0})=1\),\(\rho_{0}=\rho_{m_{0}}+\rho_{r_{0}} \approx \rho_{m_{0}} \):
  \[
    \begin{aligned}
      H_{0}^{2}&=\frac{8\pi G}{3}\rho_{m_{0}}
      \\
      t_{\alpha}&=H_{0}^{-1}\int_{0}^{a_{\alpha}}\frac{ada}{(a_{\alpha}+a)^{\frac{1}{2}}}
      \\
      &=\frac{2}{3}(2\sqrt{2}-1)H_{0}^{-1}a_{\alpha}^{\frac{3}{2}}
      \\
      & \approx 3.7 \times 10^{5} yr
    \end{aligned}
  \]
\end{homeworkProblem}

\end{document}
